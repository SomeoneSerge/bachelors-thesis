\begin{center}
  ОТЗЫВ\\
  научного руководителя на бакалаврскую работу\\
  Козлукова Сергея Викторовича\\
  \textbf{``СПЕК\-ТРАЛЬНЫЕ СВОЙ\-СТВА НЕКОТОРЫХ ВОЗМУЩ\-ЁННЫХ МАТРИЦ''},
\end{center}

Работа посвящена спектральным свойствам возмущённых матриц
специального вида, возникающих, в том числе, при исследовании свойств некоторых
графов. Отметим важность исследования спектральной теории матриц орграфов~---%
подтверждением этому являются монографии~\cite{cvetkovic1997eigenspaces,cvetkovic1980spectra,godsil2013algebraic}.
В основе исследования лежит:
\begin{itemize}
  \item Представление матрицы в виде разности двух матриц, для первой из которых
    спектр описан, а вторая мала в смысле метода подобных операторов.
    \item Метод подобных операторов, развиваемый А.~Г.~Баскаковым и его
      командой~\cite{baskakov1986theorem,baskakov1994spectral,baskakov2002splitting}.
      А именно, теорема о расщеплении линейного
      оператора.
\end{itemize}

Основные результаты составляют:

Теорема \ref{nk:thm:almost-all-ones}:
    Пусть \( M < \frac{1}{16} N^2 \),
    тогда спектр матрицы \( A \) можно представить в~виде
    объединения \( \sigma\left(A\right) = \sigma_1 \cup \sigma_2 \)
    непересекающихся
    одноэлементного множества \( \sigma_1=\{\lambda_1\} \)
    и~множества \( \sigma_2 \), удовлетворяющих условиям:
    \[ \sigma_1 \subset \left\{ \mu\in\mathbb{R}; \lvert \mu - N \rvert < 4\sqrt{M} \right\}, \]
    \[ \sigma_2 \subset \left\{ \mu\in\mathbb{C}; \lvert \mu \rvert < 4\sqrt{M} \right\}. \]

Теорема \ref{nk:thm:tiled}
Пусть \( A \) --- простого спектра, и имеет место следующее неравенство:
\[
    \left\| \mathbb{B} \right\|_{\mathrm{op}}
        \leq 
        \frac{N}{4}
         \min\left\{
             \min\limits_{1\leq i{\neq}j \leq M }{|\lambda_i - \lambda_j|},
             \min\limits_{1\leq j \leq M}{|\lambda_j|}
         \right\}.
 \]

Тогда спектр пертурбированной матрицы \( \mathbb{A} - \mathbb{B} \) представим в виде
\[
    \sigma\left(\mathbb{A}\right) =
        \left\{
            N\lambda_1 - x_{11}^o, \ldots, N\lambda_M - x_{MM}^o
        \right\}
    \cup \sigma_{M{+}1}.
\]

Собственные векторы
    \( \hat{f}_j,\ \hat{f}_{j,k},\ j{=}\overline{1,M},\ k{=}\overline{1,N{-1}} \)
    матрицы \( \mathbb{A}{-}\mathbb{B} \),
    значения \( x_{jj}^o,\ j{=}\overline{1,M} \)
    и множество \( \sigma_{M{+}1} \) удовлетворяют следующим ограничениям:
\[
    \lvert x_{jj}^o\rvert,
    \ \max_{\lambda\in\sigma_{M{+}1}} \lvert\lambda\rvert
    \leq 4\|B\|,
\]
\[
    \left\| \hat{f}_j - f_j \right\|_2,
    \ \left\| \hat{f}_{j,k} - f_{j,k}\right\|_2
    \leq
    \frac4N \|B\|
         \max\left\{
         \frac{1}{
             \min\limits_{1\leq l{\neq}p \leq M }{|\lambda_l - \lambda_p|}},
         \frac{1}{
             \min\limits_{1\leq l \leq M}{|\lambda_l|}}
         \right\}
\]
Для всех \( j{=}\overline{1,M}, k{=}\overline{1,N-1} \).
    
Вторую часть работы составляют аналогичные оценки для возмущений Кронекеровых
произведений матриц, возникающих в связи с определёнными операциями на
графах~\cite{bellman-matrices-kron,XIANG2005210}, а именно

Теорема \ref{nk:thm:kron}:
    Рассмотрим возмущённую матрицу~\eqref{-kronperturb}:
        \[
            A{\otimes}B - F.
        \]
    Пусть \( A\in\mathtt{Mat}_{N{\times}N}(\mathbb{K}) \) and \( B\in\mathtt{Mat}_{M{\times}M}(\mathbb{K}) \)
    --- диагонализуемые матрицы.
    Пусть \( f_1, \ldots, f_N \) --- eigen-векторы \( A \),
        которые соответствуют eigen-числам \( \mu_1, \ldots, \mu_N \),
        и пусть \( h_1, \ldots, h_M \) --- собственные векторы \( B \)
        с собственными числами \( \lambda_1, \ldots, \lambda_M \).
    Спектр их произведения Кронекера --- \( A{\otimes}B \) ---
        состоит из всевозможных попарных произведений \( \mu_i \lambda_j \)
        с соответствующими собственными векторами \( f_i\otimes h_j \).
    Сгруппируем эти \( MN \) произведений в \( s \) попарно различных
    собственных чисел:
        \( \nu_1, \ldots \nu_s \).

    Пусть
    \[
        \|F\| \leq \frac14 \gamma^{-1} = \frac14 \min_{1\leq i{\neq}j\leq s}\lvert\nu_i - \nu_j\rvert.
    \]

    Тогда матрица \( A{\otimes}B - F \) подобна матрице
    \[ \sum_{k=1}^s \nu_k P_k - JX^o = \sum_{k=1}^s (\nu_k P_k - P_k X^o P_k) \]
    для некоторого \( X^o \in \mathtt{Mat}_{MN{\times}MN}(\mathbb{K}) \),
    \( \|X^o - F\|\leq 3\|F\| \).

    Все собственные значения \( A{\otimes}B - F \) содержатся в шарах
    \[
        \Omega_k = \left\{
            \lambda\in\mathbb{C};
            \ \lvert\lambda - \nu_k\rvert \leq 4\|F\|
            \right\},
        \ k{=}\overline{1,s}.
    \]
    В каждом из шаров есть хотя бы одно собственное число.

    Пусть \( \nu_k=\mu_{i_k}\lambda_{j_k} \) --- собственное число \( A{\otimes}B \) кратности \( 1 \),
        id est ему соответствует единственный собственный вектор \( v_k = f_{i_k}{\otimes}h_{j_k} \).
    Эквивалентно: eigenvalue \( \mu_{i_k} \)
        матрицы \( A \) и eigenvalue \( \lambda_{j_k} \) матрицы \( B \)
        оба имеют кратность \( 1 \).
    Тогда \( A{\otimes}B - F \) имеет собственно значение в шаре \( \Omega_k \)
        которому соответствует собственный вектор \( \hat{v}_k \),
        удовлетворяющий неравенству
    \[
        \|\hat{v}_k - v_k\| \leq 4\gamma \|F\|.
    \]
    Если \( \nu_k \) хорошо отделено от прочего спектра \( A{\otimes}B \):
    \[
        \min_{l\neq k}
        \lvert
        \nu_k - \nu_l
        \rvert
        \geq 4\|F\|,\ \text{(спектральная отделимость)},
    \]
    то \( \nu_k \) --- единственное собственное значение \( A{\otimes}B - F \)
    в этом шаре.

Основными результатами работы являются локализация собственных значений и оценки
собственных векторов таких матриц.

Все полученные результаты являются новыми и были опубликованы в двух статьях в
рецензируемых журналах, один из которых входит в перечь ВАК, а другой
индексируется в Scopus и Web of Science. Работа также была представлена на двух
конференциях, индексируемых в РИНЦ.

Резюмируя изложенное, считаю, что бакалаврская работа
``СПЕК\-ТРАЛЬНЫЕ СВОЙСТВА НЕКОТОРЫХ ВОЗМУЩЁННЫХ МАТРИЦ''
безусловно заслуживает оценки ``отлично''.
Автору работы рекомендую продолжить исследования в этом направлении, как в
перспективном.

\vfill
\begin{minipage}{12em}
Доктор физ.-мат. наук, профессор,
факультета прикладной математики, информатики и механики,
проф. каф. системного анализа и управления
\end{minipage}
\hfill
проф.~А.~Г.~Баскаков
