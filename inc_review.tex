\begin{center}
  ОТЗЫВ\\
  научного руководителя на бакалаврскую работу\\
  Козлукова Сергея Викторовича\\
  \textbf{``SPECTRAL PROPERTIES OF CERTAIN PERTURBED BLOCK MATRICES''},
\end{center}

Работа посвящена спектральным свойствам возмущённых матриц
специального вида, возникающих, в том числе, при исследовании свойств некоторых
графов. Отметим важность исследования спектральной теории матриц орграфов~---%
подтверждением этому являются монографии~\cite{cvetkovic1997eigenspaces,cvetkovic1980spectra,godsil2013algebraic}.
В основе исследования лежит:
\begin{itemize}
  \item Представление матрицы в виде разности двух матриц, для первой из которых
    спектр описан, а вторая мала в смысле метода подобных операторов.
    \item Метод подобных операторов, развиваемый А.~Г.~Баскаковым и его
      командой~\cite{baskakov1986theorem,baskakov1994spectral,baskakov2002splitting}.
      А именно, теорема о расщеплении линейного
      оператора.
\end{itemize}

Основные результаты составляют:

Теорема \ref{nk:thm:almost-all-ones}:
    Пусть \( M < \frac{1}{16} N^2 \),
    тогда спектр матрицы \( A \) можно представить в~виде
    объединения \( \sigma\left(A\right) = \sigma_1 \cup \sigma_2 \)
    непересекающихся
    одноэлементного множества \( \sigma_1=\{\lambda_1\} \)
    и~множества \( \sigma_2 \), удовлетворяющих условиям:
    \[ \sigma_1 \subset \left\{ \mu\in\mathbb{R}; \lvert \mu - N \rvert < 4\sqrt{M} \right\}, \]
    \[ \sigma_2 \subset \left\{ \mu\in\mathbb{C}; \lvert \mu \rvert < 4\sqrt{M} \right\}. \]

Теорема \ref{nk:thm:tiled}
Если \( A\in\mathtt{Mat}_n \) --- простой структуры и имеет место
\[
    \left\| \mathbb{B} \right\|_{\mathrm{op}}
        \leq 
        \frac{N}{4}
         \min\left\{
             \min\limits_{1\leq i{\neq}j \leq M }{|\lambda_i - \lambda_j|},
             \min\limits_{1\leq j \leq M}{|\lambda_j|}
         \right\}.
 \]

То спектр возмущённой матрицы \( \mathbb{A} - \mathbb{B} \) представим в виде:
\[
    \sigma\left(\mathbb{A}\right) =
        \left\{
            N\lambda_1 - x_{11}^o, \ldots, N\lambda_M - x_{MM}^o
        \right\}
    \cup \sigma_{M{+}1}.
\]

The eigenvectors
    \( \hat{f}_j,\ \hat{f}_{j,k},\ j{=}\overline{1,M},\ k{=}\overline{1,N{-1}} \)
    of the matrix \( \mathbb{A}{-}\mathbb{B} \),
    the values \( x_{jj}^o,\ j{=}\overline{1,M} \)
    and the set \( \sigma_{M{+}1} \) are in the following bounds:
\[
    \lvert x_{jj}^o\rvert,
    \ \max_{\lambda\in\sigma_{M{+}1}} \lvert\lambda\rvert
    \leq 4\|B\|,
\]
\[
    \left\| \hat{f}_j - f_j \right\|_2,
    \ \left\| \hat{f}_{j,k} - f_{j,k}\right\|_2
    \leq
    \frac4N \|B\|
         \max\left\{
         \frac{1}{
             \min\limits_{1\leq l{\neq}p \leq M }{|\lambda_l - \lambda_p|}},
         \frac{1}{
             \min\limits_{1\leq l \leq M}{|\lambda_l|}}
         \right\}
\]
for all \( j{=}\overline{1,M}, k{=}\overline{1,N-1} \).
    
Вторую часть работы составляют аналогичные оценки для возмущений Кронекеровых
произведений матриц, возникающих в связи с определёнными операциями на
графах~\cite{bellman-matrices-kron,XIANG2005210}, а именно

Теорема \ref{nk:thm:kron}:
    Consider the perturbed matrix~\eqref{-kronperturb}
        \[
            A{\otimes}B - F.
        \]
    Let \( A\in\mathbb{F}^{N{\times}N} \) and \( B\in\mathbb{F}^{M{\times}M} \)
        be diagonalizable matrices.
    Let \( f_1, \ldots, f_N \) be the eigenvectors of \( A \)
        corresponding to the eigenvalues \( \mu_1, \ldots, \mu_N \)
        and let \( h_1, \ldots, h_M \) be the eigenvectors of \( B \)
        corresponding to the eigenvalues \( \lambda_1, \ldots, \lambda_M \).
    The spectrum of their Kronecker product \( A{\otimes}B \)
        is composed of all the possible pairwise products \( \mu_i \lambda_j \)
        and the corresponding eigenvectors are \( f_i\otimes h_j \).
    Suppose that out of these \( MN \) eigenvalues only \( s \) are distinct:
        \( \nu_1, \ldots \nu_s \).

    Suppose
    \[
        \|F\| \leq \frac14 \gamma^{-1} = \frac14 \min_{1\leq i{\neq}j\leq s}\lvert\nu_i - \nu_j\rvert.
    \]

    Then \( A{\otimes}B - F \) is similar to
    \[ \sum_{k=1}^s \nu_k P_k - JX^o = \sum_{k=1}^s (\nu_k P_k - P_k X^o P_k) \]
    for some \( X^o \in \mathbb{F}^{MN{\times}MN} \),
    \( \|X^o - F\|\leq 3\|F\| \).

    All the eigenvalues of \( A{\otimes}B - F \) are contained in the circles
    \[
        \Omega_k = \left\{
            \lambda\in\mathbb{C};
            \ \lvert\lambda - \nu_k\rvert \leq 4\|F\|
            \right\},
        \ k{=}\overline{1,s}.
    \]
    There is at least one eigenvalue in each of these circles.

    Suppose the eigenvalue \( \nu_k=\mu_{i_k}\lambda_{j_k} \) of \( A{\otimes}B \) has multiplicity \( 1 \),
        that is it has the only eigenvector \( v_k = f_{i_k}{\otimes}h_{j_k} \).
    It is equivalent to the statement that the eigenvalue \( \mu_{i_k} \)
        of \( A \) and the eigenvalue \( \lambda_{j_k} \) of \( B \)
        are both of multiplicity \( 1 \).
    Then \( A{\otimes}B - F \) has eigenvalue in the circle \( \Omega_k \)
        and the corresponding eigenvector \( \hat{v}_k \) is within bounds
    \[
        \|\hat{v}_k - v_k\| \leq 4\gamma \|F\|.
    \]
    If \( \nu_k \) is well separated from all the other eigenvalues of \( A{\otimes}B \):
    \[
        \min_{l\neq k}
        \lvert
        \nu_k - \nu_l
        \rvert
        \geq 4\|F\|,
    \]
    then \( \nu_k \) is the only eigenvalue of \( A{\otimes}B - F \)
    in that circle.

Основными результатами работы являются локализация собственных значений и оценки
собственных векторов таких матриц.

Все полученные результаты являются новыми и были опубликованы в двух статьях в
рецензируемых журналах, один из которых входит в перечь ВАК, а другой
индексируется в Scopus и Web of Science. Работа также была представлена на двух
конференциях, индексируемых в РИНЦ.

Резюмируя изложенное, считаю, что бакалаврская работа
``SPECTRAL PROPERTIES OF
CERTAIN PERTURBED BLOCK MATRICES''
безусловно заслуживает оценки ``отлично''.
Автору работы рекомендую продолжить исследования в этом направлении, как в
перспективном.

\vfill
\begin{minipage}{12em}
Доктор физ.-мат. наук, профессор,
факультет прикладной математики, информатики и механики,
кафедра нелинейных колебаний
\end{minipage}
\hfill
А.~Г.~Баскаков
