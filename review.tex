\documentclass[a4paper]{article}

\usepackage[russian,english]{babel}
\usepackage{fontspec}
\setmainfont{CMU Serif}

\begin{document}
\begin{center}
  ОТЗЫВ\\
  научного руководителя на бакалаврскую дипломную работу\\
  Козлукова Сергея Викторовича\\
  \textbf{``SPECTRAL PROPERTIES OF CERTAIN PERTURBED BLOCK MATRICES''},\\
  01.01.01 -- вещественный, комплексный и функциональный анализ\textbf{?}
\end{center}

Дипломная работа посвящена спектральным свойствам возмущённых матриц
специального вида, возникающих, в том числе, при исследовании свойств некоторых
графов. Наиболее наглядный пример рассмотренного класса матриц --- матрица
смежностей почти-полного орграфа,
все элементы которой --- единицы, кроме небольшого числа нулей.
В общем случае рассмотрены матрица, составленная из одинаковых квадратных
блоков, а также Кронекеровы произведения квадратных матриц, соответствующие
матрицам смежностей NEPS сумм графов.
Основной инструмент исследования --- метод подобных операторов.

Основными результатами работы являются локализация собственных значений и оценки
собственных векторов таких матриц.

Основные результаты являются новыми и строго обоснованными
\textbf{somethang in the field of somewhat}.

Замечу, что...

Результаты дипломной работы своевременно опубликованы в четырёх работах, из них
две (одна --- Волгоград?) опубликованы в журналах из перечня рецензируемых
научных журналов и  изданий, рекомендованных ВАК Минобрнауки РФ.
Отметим, что одна из них (в журнале ``Journal of Physics: Conference Series'')
цитируется в Web of Science и Scopus.

Резюмируя изложенное, считаю, что дипломная работа
``SPECTRAL PROPERTIES OF
CERTAIN PERTURBED BLOCK MATRICES''
удовлетворяет всем требованиям \textbf{de quelqu'un}, предъявляемым к
\textbf{something} по специальности 01.01.01 --- вещественный, комплексный и
функциональный анализ, а её автор Козлуков Сергей Викторович заслуживает
\textbf{etwas}.


\vfill
\begin{minipage}{12em}
Доктор физ.-мат. наук, профессор,
факультет прикладной математики, информатики и механики,
кафедра нелинейных колебаний
\end{minipage}
\hfill
А.~Г.~Баскаков
\end{document}