%\documentclass[a4paper]{extarticle}
\documentclass[14pt,a4paper]{extarticle}

\sloppy

\usepackage[left=30mm,right=20mm,top=25mm,bottom=20mm]{geometry}
\usepackage{marginnote}
\usepackage{hyperref}
\usepackage{tabularx}

% ----
\usepackage[russian,english]{babel}
\usepackage{fontspec}
\setmainfont{CMU Serif}
% \setmainfont[SizeFeatures={Size=14}]{CMU Serif}


\usepackage{amsmath,amsthm,amssymb,euscript,calc,graphicx,ifthen,textcomp,rotating,cmap,xstring,etoolbox,xparse,l3regex,xargs,changepage,tocloft,titlecaps,enumitem,upgreek}
\usepackage{mathrsfs}

\numberwithin{equation}{section}

\newtheorem{thm}{Теорема}
\newtheorem*{thm*}{Теорема}
\newtheorem{lem}{Лемма}
\newtheorem{crl}{Следствие}
\theoremstyle{definition}


\usepackage[dotinlabels]{titletoc}
\usepackage[center]{titlesec}
\titlelabel{\thetitle.\quad}

\usepackage{fancyhdr}
\fancyhf{} % clear all header and footers
\renewcommand{\headrulewidth}{0pt} % remove the header rule
\chead{\thepage}
\pagestyle{fancy}


\renewcommand{\contentsname}{Содержание}
\addto\captionsenglish{% Replace "english" with the language you use
  \renewcommand{\contentsname}%
    {Содержание}%
}

\usepackage[backend=biber,
bibstyle=gost-numeric,
sorting=ntvy,
]{biblatex}
\addbibresource{thesis.bib}
\renewcommand{\baselinestretch}{1.3}

\begin{document}

\begin{center}
  ОТЗЫВ\\
  научного руководителя на бакалаврскую работу\\
  Козлукова Сергея Викторовича\\
  \textbf{``СПЕК\-ТРАЛЬНЫЕ СВОЙ\-СТВА НЕКОТОРЫХ ВОЗМУЩ\-ЁННЫХ МАТРИЦ''},
\end{center}

Работа посвящена спектральным свойствам возмущённых матриц
специального вида, возникающих, в том числе, при исследовании свойств некоторых
графов. Отметим важность исследования спектральной теории матриц орграфов~---%
подтверждением этому являются монографии~\cite{cvetkovic1997eigenspaces,cvetkovic1980spectra,godsil2013algebraic}.
В основе исследования лежит:
\begin{itemize}
  \item Представление матрицы в виде разности двух матриц, для первой из которых
    спектр описан, а вторая мала в смысле метода подобных операторов.
    \item Метод подобных операторов, развиваемый А.~Г.~Баскаковым и его
      командой~\cite{baskakov1986theorem,baskakov1994spectral,baskakov2002splitting}.
      А именно, теорема о расщеплении линейного
      оператора.
\end{itemize}

Основные результаты составляют:

Теорема \ref{nk:thm:almost-all-ones}:
    Пусть \( M < \frac{1}{16} N^2 \),
    тогда спектр матрицы \( A \) можно представить в~виде
    объединения \( \sigma\left(A\right) = \sigma_1 \cup \sigma_2 \)
    непересекающихся
    одноэлементного множества \( \sigma_1=\{\lambda_1\} \)
    и~множества \( \sigma_2 \), удовлетворяющих условиям:
    \[ \sigma_1 \subset \left\{ \mu\in\mathbb{R}; \lvert \mu - N \rvert < 4\sqrt{M} \right\}, \]
    \[ \sigma_2 \subset \left\{ \mu\in\mathbb{C}; \lvert \mu \rvert < 4\sqrt{M} \right\}. \]

Теорема \ref{nk:thm:tiled}
Пусть \( A \) --- простого спектра, и имеет место следующее неравенство:
\[
    \left\| \mathbb{B} \right\|_{\mathrm{op}}
        \leq 
        \frac{N}{4}
         \min\left\{
             \min\limits_{1\leq i{\neq}j \leq M }{|\lambda_i - \lambda_j|},
             \min\limits_{1\leq j \leq M}{|\lambda_j|}
         \right\}.
 \]

Тогда спектр пертурбированной матрицы \( \mathbb{A} - \mathbb{B} \) представим в виде
\[
    \sigma\left(\mathbb{A}\right) =
        \left\{
            N\lambda_1 - x_{11}^o, \ldots, N\lambda_M - x_{MM}^o
        \right\}
    \cup \sigma_{M{+}1}.
\]

Собственные векторы
    \( \hat{f}_j,\ \hat{f}_{j,k},\ j{=}\overline{1,M},\ k{=}\overline{1,N{-1}} \)
    матрицы \( \mathbb{A}{-}\mathbb{B} \),
    значения \( x_{jj}^o,\ j{=}\overline{1,M} \)
    и множество \( \sigma_{M{+}1} \) удовлетворяют следующим ограничениям:
\[
    \lvert x_{jj}^o\rvert,
    \ \max_{\lambda\in\sigma_{M{+}1}} \lvert\lambda\rvert
    \leq 4\|B\|,
\]
\[
    \left\| \hat{f}_j - f_j \right\|_2,
    \ \left\| \hat{f}_{j,k} - f_{j,k}\right\|_2
    \leq
    \frac4N \|B\|
         \max\left\{
         \frac{1}{
             \min\limits_{1\leq l{\neq}p \leq M }{|\lambda_l - \lambda_p|}},
         \frac{1}{
             \min\limits_{1\leq l \leq M}{|\lambda_l|}}
         \right\}
\]
Для всех \( j{=}\overline{1,M}, k{=}\overline{1,N-1} \).
    
Вторую часть работы составляют аналогичные оценки для возмущений Кронекеровых
произведений матриц, возникающих в связи с определёнными операциями на
графах~\cite{bellman-matrices-kron,XIANG2005210}, а именно

Теорема \ref{nk:thm:kron}:
    Рассмотрим возмущённую матрицу~\eqref{-kronperturb}:
        \[
            A{\otimes}B - F.
        \]
    Пусть \( A\in\mathtt{Mat}_{N{\times}N}(\mathbb{K}) \) and \( B\in\mathtt{Mat}_{M{\times}M}(\mathbb{K}) \)
    --- диагонализуемые матрицы.
    Пусть \( f_1, \ldots, f_N \) --- eigen-векторы \( A \),
        которые соответствуют eigen-числам \( \mu_1, \ldots, \mu_N \),
        и пусть \( h_1, \ldots, h_M \) --- собственные векторы \( B \)
        с собственными числами \( \lambda_1, \ldots, \lambda_M \).
    Спектр их произведения Кронекера --- \( A{\otimes}B \) ---
        состоит из всевозможных попарных произведений \( \mu_i \lambda_j \)
        с соответствующими собственными векторами \( f_i\otimes h_j \).
    Сгруппируем эти \( MN \) произведений в \( s \) попарно различных
    собственных чисел:
        \( \nu_1, \ldots \nu_s \).

    Пусть
    \[
        \|F\| \leq \frac14 \gamma^{-1} = \frac14 \min_{1\leq i{\neq}j\leq s}\lvert\nu_i - \nu_j\rvert.
    \]

    Тогда матрица \( A{\otimes}B - F \) подобна матрице
    \[ \sum_{k=1}^s \nu_k P_k - JX^o = \sum_{k=1}^s (\nu_k P_k - P_k X^o P_k) \]
    для некоторого \( X^o \in \mathtt{Mat}_{MN{\times}MN}(\mathbb{K}) \),
    \( \|X^o - F\|\leq 3\|F\| \).

    Все собственные значения \( A{\otimes}B - F \) содержатся в шарах
    \[
        \Omega_k = \left\{
            \lambda\in\mathbb{C};
            \ \lvert\lambda - \nu_k\rvert \leq 4\|F\|
            \right\},
        \ k{=}\overline{1,s}.
    \]
    В каждом из шаров есть хотя бы одно собственное число.

    Пусть \( \nu_k=\mu_{i_k}\lambda_{j_k} \) --- собственное число \( A{\otimes}B \) кратности \( 1 \),
        id est ему соответствует единственный собственный вектор \( v_k = f_{i_k}{\otimes}h_{j_k} \).
    Эквивалентно: eigenvalue \( \mu_{i_k} \)
        матрицы \( A \) и eigenvalue \( \lambda_{j_k} \) матрицы \( B \)
        оба имеют кратность \( 1 \).
    Тогда \( A{\otimes}B - F \) имеет собственно значение в шаре \( \Omega_k \)
        которому соответствует собственный вектор \( \hat{v}_k \),
        удовлетворяющий неравенству
    \[
        \|\hat{v}_k - v_k\| \leq 4\gamma \|F\|.
    \]
    Если \( \nu_k \) хорошо отделено от прочего спектра \( A{\otimes}B \):
    \[
        \min_{l\neq k}
        \lvert
        \nu_k - \nu_l
        \rvert
        \geq 4\|F\|,\ \text{(спектральная отделимость)},
    \]
    то \( \nu_k \) --- единственное собственное значение \( A{\otimes}B - F \)
    в этом шаре.

Основными результатами работы являются локализация собственных значений и оценки
собственных векторов таких матриц.

Все полученные результаты являются новыми и были опубликованы в двух статьях в
рецензируемых журналах, один из которых входит в перечь ВАК, а другой
индексируется в Scopus и Web of Science. Работа также была представлена на двух
конференциях, индексируемых в РИНЦ.

Резюмируя изложенное, считаю, что бакалаврская работа
``СПЕК\-ТРАЛЬНЫЕ СВОЙСТВА НЕКОТОРЫХ ВОЗМУЩЁННЫХ МАТРИЦ''
безусловно заслуживает оценки ``отлично''.
Автору работы рекомендую продолжить исследования в этом направлении, как в
перспективном.

\vfill
\begin{minipage}{12em}
Доктор физ.-мат. наук, профессор,
факультета прикладной математики, информатики и механики,
проф. каф. системного анализа и управления
\end{minipage}
\hfill
проф.~А.~Г.~Баскаков

\newpage

\makeatletter
{
  \null
  \thispagestyle{empty}
  \newpage
}

\setcounter{page}{0}
\begin{center}
  МИНОБРНАУКИ РОССИИ\\
  ФЕДЕРАЛЬНОЕ ГОСУДАРСТВЕННОЕ БЮДЖЕТНОЕ ОБРАЗОВАТЕЛЬНОЕ УЧРЕЖДЕНИЕ ВЫСШЕГО
  ОБРАЗОВАНИЯ\\ ``ВОРОНЕЖСКИЙ ГОСУДАРСТВЕННЫЙ УНИВЕРСИТЕТ''\\
(ФГБОУ ВО ``ВГУ'')
\end{center}

\vfill

\begin{center}
  Факультет Прикладной математики, информатики и механики\\
  Кафедра системного анализа и управления
\end{center}

\vfill

\begin{center}
  \textsc{\Large Спектральные свойства некоторых возмущённых матриц}\\[.5cm]
  Выпускная квалификационная работа\\
  Направление 01.03.02 Прикладная математика и информатика\\
  Профиль Нелинейная динамика
\end{center}

\vfill
\begin{center}
  {Допущена к защите в ГЭК 04.06.2018}
  \end{center}

\vfill
\begin{center}
  {Зав. кафедрой \hfill д.ф.-м.н.,~профессор В.~Г.~Задорожний}
  \vspace{1cm}

  {Обучающийся \hfill С.~В.~Козлуков}

  \vspace{1cm}
  {Научный руководитель \hfill д.ф.-м.н.,~профессор~А.~Г.~Баскаков}
\end{center}
 
\vfill
\begin{center}
  Воронеж~2018
\end{center}

{
\pagestyle{empty}
\newpage
}
\setcounter{tocdepth}{2}
\tableofcontents
\newpage

\section{Нотация}

В работе используются следующие обозначения и соглашения

\( f: X\to Y: x\mapsto f(x) \) --- функция \( f \)
из domain множества \( X \) в target множество \( Y \),
сопоставляющая каждому \( x\in X \) некоторый \( y=f(x) \in Y\);

\( \mathtt{card}X \) --- кардинальное число множества \( X \);

\( \mathbb{N} = \{ 1, 2, \ldots \}\) --- множество натуральных чисел без нуля;

\( \mathbb{Z} \) --- кольцо целых чисел;

\( \mathbb{R} \) --- поле вещественных чисел.

Промежутки вещественной прямой обозначаются пределами, раз\-дел\-ёнными двумя
точками так, чтобы отличать интервалы  ---
\( (a..b] \subset \mathbb{R} \) --- от упорядоченных пар: \( (a, b)\in\mathbb{R}^2 \).

\( [a..b] = \{ x\in\mathbb{R}:\ a\leq x\leq b\} \) --- отрезок
от \( a \) до \( b \), \( a\leq b \);

\( [a..b) = \{ x\in\mathbb{R}:\ a\leq x < b\} \) --- полуинтервал;

\( (a..b) = \{ x\in\mathbb{R}:\ a < x < b\} \) --- открытый интервал;

\( \mathbb{R}_+ = [0..+\infty) \) --- множество неотрицательных вещественных чисел;

\( \mathbb{C} \) --- поле комплексных чисел;

\( \mathbb{K} \) --- одно из полей: \( \mathbb{R} \) или \( \mathbb{C} \);

\( M{\times}N = \{ (m, n):\ m{\in}M,\ n{\in}N \} \), где \( M \) и \( N \)
--- множества, обозначает прямое произведение этих множеств;

\( I \) означает тождественный оператор, а \( E \) единичную матрицу;

\( \sigma(A) \) --- спектр матрицы или оператора \( A \);

\( \rho(A) \) --- резольвентное множество матрицы (оператора) \( A \);

\( \mathtt{Hom}(\mathscr{X}, \mathscr{Y})\) --- Банахово пространство ограни\-чен\-ных
линейных операторов, определённых на \( \mathscr{X} \) с значениями в \( \mathscr{Y} \).

\( \mathtt{End}\mathscr{X} \) --- Банахова алгебра операторов на
\( \mathscr{X} \);

\( \mathtt{Mat}_{m{\times}n}(\mathbb{K}) \) --- Банахово пространство матриц
размера \( m\times n \) с естественными умножением матриц
совместных размеров;

\( \mathtt{Mat}_{n}(\mathbb{K}) = \mathtt{Mat}_{n{\times}n}(\mathbb{K}) \) ---
Банахова алгебра квадратных матриц размера \( n{\times}n \);


\newpage

\section{Введение}
Работа посвящена исследованию спектральных свойств определённого класса, или,
вернее, классов матриц,
особенно специальных матриц, возникающих в отношении теории возмущения
графов~\cite{cvetkovic1997eigenspaces}.

\textbf{Целью работы}
является локализация собственных значений и собственных векторов
матриц специального вида.

Основным \textbf{методом исследования} является ме\-тод подобных опер\-атор\-ов,
развиваемый~\cite{baskakov1986theorem,baskakov1987theorem,baskakov1994spectral,baskakov2002splitting}
А.~Г.~Баскаковым и его командой. Именно, используется представление матрицы
в виде разности хорошо изученной части и пренебрежимо-малого возмущения.
В частности, мы рас\-сматр\-иваем матрицу смежностей
``почти-полного орграфа''~\cite{Koz17,sergekozlukov@vspu},
все элементы которой за небольшим числом --- единицы;
рассматриваем возмущения аналогичной матрицы, составленной из одинаковых
квадрат\-ных блоков --- здесь мы её называем ``a tiled
matrix''~\cite{Koz18,sergekozlukov@currentproblems};
наконец, мы обобщ\-аем результат на случай (возмущений) Кронекер\-овых
произведений квадратных матриц
~\cite{Koz18,sergekozlukov@currentproblems,bellman-matrices-kron,XIANG2005210}.
Заметим, что последние возникают в связи с некоторыми операциями на графах~\cite{cvetkovic1997eigenspaces}.

В ходе работы был получен ряд новых результатов, представленных на
конференциях~\cite{sergekozlukov@vspu,sergekozlukov@currentproblems} и
опубликованных в статьях~\cite{Koz17,Koz18}.

\textbf{Структура и объём работы}. Работа состоит из введения, трёх секций с
представлением основных результатов, и би\-блио\-графии, со\-став\-ляющей 29
на\-имено\-ва\-ний. Общий объём --- 33 страницы.


\subsection{Элементы теории графов}

Под графом мы понимаем тройку \((V, E, \phi)\), состоящую
из множества \( V \) вершин, множества \( E \) рёбер,
и отображения \( \phi: E\to V^2\). Где не означено иное,
множество \( V \) будем считать конечным и пронумерованным: \( V = \{ v_1, v_2,
\ldots, v_n \}\). 
Будем говорить, что ребро \( e \) \emph{исходит из} (\emph{is originating at})
вершины \( v_i\in V \)
и входит (\emph{is terminating at}) в вершину \( v_j\in V \),
если \( \phi(e)=(v_i, v_j)\). При этом вершины \( v_i, v_j \) называются
\emph{смежными} по отношению друг к другу, и \emph{инцидентными} к ребру \( e \).
На практике мы будем считать кратные рёбра неразличимыми,
так что граф однозначно задаётся его \emph{матрицей смежностей}
\[ A = ( a_{ij} ) \in \mathtt{Mat}_n(\mathbb{R}), \]
\[ a_{ij} = \mathtt{card}\{ e: \phi(e)=(v_i, v_j)\}, \]
где \( \mathtt{Mat}_n(\mathbb{R}) \) --- Банахова алгебра всех
вещественно-значных матриц размера
\( n{\times}n \) с естественной операцией умножения.
Значение \( \lambda \in \mathbb{C} \) называют собственным значением
матрицы \(
A\in\mathtt{Mat}_n(\mathbb{R}) \) если матрица \( A - \lambda E \) с \( E \) --- единичной
матрицей, необратима. Множество всех таких значений называется
\emph{спектром} матрицы \( A \) и обозначается \( \sigma(A) \).
При фиксированном \( \lambda \),  множество всех векторов \( h\in\mathbb{C}^n \)
для которых \( A h = \lambda h \) называется собственным подпространством матрицы
\( A \), соответствующим собственному значению \( \lambda \).
Спектры и собственные подпространства матриц смежностей несут в себе важную
информацию о соответствующих графах. Однако, точное вы\-числ\-ение собств\-енных
значений и век\-торов на практике чрез\-вычайно ресурсо\-ёмкая задача, страдающая
своего рода \emph{curse of dimensionality}. Поэтому требуются дешёвые и
дос\-таточно точные оценки.

Различные спектры графов определяются с помощью различ\-ных матриц,
соответ\-ствующих этим графам.
Так называемый \emph{simple spectrum} --- простой спектр --- графа
определяется как спектр матрицы смежностей.
В прикладных моделях часто требуется рас\-сматривать комбинации
    матриц \( A \),
    \( D \) (``out-degree'' матрица, диагональные элементы которой
    содержат исходящие степени вершин),
    \( E \) (единичная матрица)
    и матрицы \( \mathcal{J}_N \) (матрица единиц).
Такие комбинации естественным образом возникают во многих стохастич\-еских
моделях~\cite[p.~184]{cvetkovic2010introduction}.
Спектральные свойства таких матриц
часто играют существенную роль в этих моделях.
Например, Марковское случайное блуждание на графе
    приводит к понятию т.н. ``eigenvector centrality''
    в сети~\cite{ilprints422,bonacich1972factoring}.
Eigenvector-ранг \( i \)-й вершины определяется как
\( i \)-я координата
доминирующего (т.е. соответствующего наибольшему по модулю собственному
значению) (левого) собственного вектора 
матрицы переходов рас\-сматр\-иваемого процесса.
Этот собственный вектор
    задаёт (единственное) стационарное распределение этого процесса.
Алгоритм PageRank~\cite{ilprints422}
    изначально лежавший в основе поискового движка Google
    использует метод power-итерации для вы\-числ\-ения глав\-но\-го собствен\-ного
    вектора.
Скорость сходимости метода зависит от
    отношения двух наибольших сосбтвенных значений.
У\-стой\-чив\-ость ста\-цио\-нар\-ного рас\-пред\-еления
    опр\-еделена~\cite{meyer1994sensitivity}
    кон\-ди\-цион\-ным числом (числом обу\-слов\-лен\-ности) матрицы пере\-ходов про\-цесса,
    которое снизу ограни\-чено спек\-траль\-ным зазором
    --- рас\-стояние между двумя наи\-большими по модулю собст\-вен\-ными зна\-чениями
    этой матрицы.
Метод ап\-прокс\-имации почти-инвариант\-ных множеств, описанный в~\cite{schwartz2006fluctuation}
также опирается на спек\-тральное раз\-ложение мат\-рицы пере\-ходов.
В Suscept\-ible-\-Infective-\-Suscept\-ible модели
рас\-пространения~вируса в сети
модели\-руется~\cite{wang2003epidemic,chakrabarti2008epidemic} Марков\-ским про\-цессом
на \( 2^N \) состояниях --- состояние сети задаётся век\-тором состояний всех \( N
\) вершин . Это discrete-time модель, в~которой спек\-тральный радиус
матрицы смеж\-ностей графа сети  оказы\-вается порого\-вым значением \( {^1/_{\tau_0}}
\) отношения \( {^1/_\tau = ^\delta/_\nu} \) интенсивн\-ости~\( \delta \) исцеления
ин\-фе\-цир\-ован\-ных узлов и~интенсив\-ности~\( \nu \) заражения узлов, смежных
ин\-фе\-цир\-ован\-ным. 
А\-сим\-пто\-тическое эндеми\-ческое или эпидеми\-ческое по\-ве\-дение такой системы
опре\-деляется нахож\-дением величины \(^1/\tau\) слева или справа от
спек\-трального радиуса (т.е. наи\-больш\-его абсолютного собственного значения)
матрицы смежностей графа сети.

Многие операции на графе задаются в терминах т.н.
    \emph{non-complete extended p-sum}s~(NEPS)
    графов~\cite[p.~44]{cvetkovic2010introduction}~\cite{cvetkovic1997eigenspaces},
    которые в свою очередь задаются в терминах Кронекеровых произве\-де\-ний (степеней) матриц
    смежностей графов.

Теория спектров и собственных подпространств графов,
    а также теория возмущений графов, подробно изложены в
    монографиях~\cite{cvetkovic1997eigenspaces,cvetkovic1980spectra,cvetkovic2010introduction}.
    За общим изложением теории графов отсылаем читателя к~\cite{godsil2013algebraic}.


\subsection{Введение в метод подобных операторов}

Как мы уже заметили выше, точное вычисление собствен\-ных чисел и векторов
для сущ\-ествен\-но больших матриц --- задачи алгоритмически сложная (i.e. ``сложная'' в
смысле time complexity). Однако часто то достаточно лишь при\-близ\-ительной оценки
--- интер\-валов (шаров), со\-держ\-ащих собствен\-ные зна\-чения (векторы).
Один из методов построения таких оценок ---
\emph{метод подобных операторов}, предлагающий как кон\-крет\-ные оценки, так и
итеративную процедуру точного вычисления собственных значений и векторов.
Истоки метода лежатв работах Fried\-richs~\cite{friedrichs1965advanced} и в
абстрактной постановке были разработаны
Баскак\-овым~А.~Г.~\cite{baskakov1986theorem,baskakov1987theorem,baskakov1994spectral,baskakov2002splitting}.
В его основе лежит понятие сжимающего отображения в Банаховом пространстве
и связанная с ним теорема о неподвижной точки of Banach's.
Этот подход часто даёт лучшие результаты в сравнении с клас\-сич\-еск\-ими методами
теории возмущений, основанными на разложе\-нии функций в степенные ряды.

В настоящем изложении мы пока ограни\-чиваемся конечно-мерным случаем, а потому
приводим ниже формулировку основных понятий и теорем метода в их соответственным
образом упрощённом --- конечно-мерном --- виде.

Пусть \( \mathbb{K}\in \{ \mathbb{R}, \mathbb{C} \} \)
--- какое-нибудь поле.
Мы рассматриваем векторное пространство \( \mathbb{K}^n,\ n\in \mathbb{N} \)
наделённым Эвклидовой структурой:
    \[
        \langle x, y\rangle{=}\sum_{k=1}^n x_k\overline{y_k},
        \ x{=}(x_1,\ldots, x_n),
        \ y=(y_1,\ldots, y_n)
        \in \mathbb{K}^n
        \]
    и порождаемой ей \( \mathrm{L}_2 \)-нормой:
    \(
        \|x\|_2^2{=}\langle x,x\rangle.
        \)
В этом пространстве мы выделяем канонический базис \( e_1, \ldots, e_n \),
        заданный формулами:
        \( {(e_i)}_j = \delta_{ij},\ i,j=\overline{1,n} \)
    (где \(\delta_{ij} \) --- ``Delta'' Кронекера).
Для нормированных пространств \( V_1, V_2 \) 
    мы будем обозначать симвлом \( \mathtt{Hom}(V_1, V_2) \)
    пространство ограни\-ченных линейных операторов
    из \( V_1 \) в \( V_2 \).
Здесь сразу следует заметить, что мы не спешим без необходимости различать
``снабжённое нормой \(x\mapsto \|x\| \) множество \( X \)''
и ``нормированное пространство \( (X, x\mapsto \|x\|) \)''
и, где это не создаёт двусмысленности, обозначаем оба символом \( X \).
Алгебра (линейных ограниченных) эндоморфизмов
    из Банахова пространства \( V \)
    в себя обозначается символом \( \mathtt{End}(V) = \mathtt{Hom}(V, V) \).
На самом деле это множество наделено ещё более структурой \emph{Банаховой
  алгебры} --- кроме операции умножения на нём определена суб\-мульти\-пликатив\-ная
(относительно этого умножения) норма:
    \[
        \|A\|_{\mathrm{op}} =
        \sup_{
            \substack{\|x\|=1,\\ x\in V}
        } \|A x\|,\ A\in \mathtt{End}(V),
        \]
     \[
       \|AB\|_{\mathrm{op}} \leq \|A\|_{\mathrm{op}} \|B\|_{\mathrm{op}}.
        \]
Вместе с пространством \( \mathtt{Hom}(\mathbb{K}^n, \mathbb{K}^m) \)
    мы рассматриваем изоморфные им пространства \(
    \mathtt{Mat}_{m{\times}n}(\mathbb{K}) \)
    матриц размера \( m{\times}n \)
    с коэффициентами из поля \( \mathbb{K} \).
Пространство \( \mathtt{Mat}_{n{\times}n}(\mathbb{K})\sim
\mathtt{End}(\mathbb{K}^n) \)
также образует алгебру Банаха, снабженное любой суб\-мульти\-пликатив\-ной нормой
    \( \|\cdot\| \):
    \( \|A\|_{\mathrm{op}} = \sup_{\|x\|_2=1,\ x\in \mathbb{K}^n} \|A x\|_2,\ \)
    \( \|A\|_{\mathrm{F}} = \sqrt{\sum_{i,j} |a_{ij}|^2},\ \)
    для 
    \( A{=}(a_{ij})\in\mathbb{K}^{n\times n} \).
Наконец, в работе возникнут изоморфные пространства
    \( \mathtt{End}(\mathtt{End}(\mathbb{K}^n)) \) и \(
    \mathtt{End}(\mathtt{Mat}_{n{\times}n}(\mathbb{K})) \)
    снабжённые соответствующими операторными нормами.
Следуя Крейну, элементы \( \mathtt{End}(\mathtt{Mat}_{n{\times}n}(\mathbb{K})) \)
    будем называть ``транс\-форм\-аторами''.

Спектр матрицы \( A \)
    (множество собственных значений)
    мы будем обозначать \( \sigma(A) \).
Две матрицы \( A_1, A_2 \) будем понимать \emph{подобными}
    в смысле существования обратимой матрицы \( U \)
    (матрицы преобразования подобия),
    такой что \( A_1 U = U A_2 \).
Подобные матрицы ``разделяют'' многие спектральные свойства:
    подобные матрицы изоспектральны, (\( \sigma(A_1) = \sigma(A_2) \)),
    а матрица подобия \( U \) отображает собственные векторы одной матрицы
    в соответствующие собственные векторы второй матрицы,
    \( A_2 x = \lambda x \implies A_1 U x = \lambda U x \).

Основным понятием абстрактного метода подобных операторов
явл\-яется понятие \emph{допустимой тройки} (\emph{an admissible triple}).
В нашем --- конечно\-мерном случае --- мы будем говорить, что
    \( (\mathtt{Mat}_{n{\times}n}(\mathbb{K}), J, \Gamma) \)
    формирует допустимую тройку
    для матрицы \( A\in\mathtt{Mat}_{n{\times}n}(\mathbb{K}) \)
    если вы\-полняются следующие условия:
\begin{itemize}
    \item \( J, \Gamma \in \mathtt{End}(\mathtt{Mat}_{n{\times}n}(\mathbb{K})) \)
      являются транс\-форм\-аторами;
    \item \( J \) --- линейный проектор (\( J^2 = J \));
    \item  \( \Gamma \) удовле\-творяет уравнениям:
        \[
            A \Gamma X - (\Gamma X) A = X - JX,
        \]
        \[
            J\Gamma X = 0,\ X\in\mathtt{Mat}_{n{\times}n}(\mathbb{K}).
        \]
\end{itemize}

Теперь мы готовы сформулировать основную теорему метода подобных операторов:

\begin{thm}
    Рассмотрим матрицу \( A - B \),
         \( A, B \in \mathtt{Mat}_{n{\times}n}(\mathbb{K}) \).
    Пусть \( (\mathtt{Mat}_{n{\times}n}(\mathbb{K}), J, \Gamma) \)
        является допустимой тройкой для \( A \)
        и выполняется неравенство:
        \[
            \|B\|\|\Gamma\|_{\mathrm{op}} \leq \frac14.
        \]

    Тогда существует такая матрица \( X^o\in\mathtt{Mat}_{n{\times}n}(\mathbb{K}) \)
        что исходная матрица \( A - B \) подобна более ``простой'' матрице \( A
        - J X^o \);
        соответ\-ствующие преобразование подобия определяется матрицей \( E +
        \Gamma X^o \);
        имеют место следующие оценки:
        \[
            \|X^o - B\| \leq 3 \left\|B\right\|,
        \]
        \[
            \operatorname{spr}(X^o) \leq \|X^o\| \leq 4 \left\|B\right\|,
        \]
        где \( \operatorname{spr}(X^o) \)
        --- спектральный радиус матрицы \( X^o \) (наибольшее по модулю
        собственное значение).
    Эта матрица \( X^o \) может быть вычислена как предел сходящейся последовательности
        \( \left( \Phi^k(0);\ k\in\mathbb{N} \right) \)
        в Банаховой алгебре \( \mathtt{Mat}_{n{\times}n}(\mathbb{K}) \).
        Здесь \( \Phi \) --- (нелинейное) сжимающее отображение
        шара \( \{X\in\mathtt{Mat}_{m{\times}n}(\mathbb{K});\ \|X-B\|\leq 3\|B\| \} \)
        в себя, определяемое формулой
    \[
        \Phi(X) = B\Gamma X - (\Gamma X)J(B + B\Gamma X) + B
    \]
        а \( \Phi^k = \underbrace{\Phi\circ\cdots\circ\Phi}_{k\ \text{copies}} \)
        означает функциональную композицию.
\end{thm}

\newpage

\section{Матрица смеж\-ностей
  почти-полного ор\-графа}
Рассмотрим орграф заданный матрицей смежностей следующего вида:
\[
    A = \mathcal{J}_N - B = \begin{pmatrix}1 & \cdots & 1 \\ \vdots & \ddots & \vdots \\ 1 & \cdots & 1\end{pmatrix} - B,
\]

где \( \mathcal{J}_N \) --- ``матрица единиц'' (an all-ones matrix),
а матрица \( B \) состоит из \( N^2 - M \) нулей и \( M \) единиц.
Матрицу \( A \) следует интерпретировать следующим образом.
Матрица \( \mathcal{J}_N \) соответствует полному графу на \( N \) вершинах, в
котором к каждой вершине ``подвесили'' петлю --- ребро из вершины в себя.
Теперь, единица на пересечении
\( i \)-й строки и \( j \)-го столбца матрица \( B \)
означает удалённое из этого графа ребро \( (i, j) \).
Этот пример был рассмотрен и опубликован в работе~\cite{sergekozlukov@volgograd},
которую мы здесь воспроизведем для демонстрации техники.

Что можно сказать о~собственных значениях матриц рассматриваемого вида?

Спектр \( \sigma\left( \mathcal{J}_N \right) \)
 матрицы \( \mathcal{J}_N \) легко считается: минимальный аннулирующий
 многочлен \( \mathcal{J}_N \) имеет вид \( \lambda(\lambda - N) \) (так как \(
 \mathcal{J}_N^2 = N \mathcal{J}_N, \)), а потому
 \[ \sigma\left( \mathcal{J}_N \right) = \left\{ 0,N \right\}. \]

Очевидно, при достаточно малых \( M \),
спектры матриц \( \mathcal{J}_N \) и~\( A \) будут ``близки''.
В терминах теории возмущений, мы понимаем матрицу \( \mathcal{J}_N \)
``идеальным случаем'', а матрицу \( B \) --- малым возмущением.
Метод подобных операторов (см.~\cite{baskakov1986theorem,baskakov1987theorem,baskakov1994spectral,baskakov2002splitting}),
предлагает систему уравнений для нахождения матрицы подобной \( \mathcal{J}_N -
B \), но имеющей более ``простую'' структуру. Эти уравнения решаются методом
простой итерации~\cite{baskakov1986theorem,baskakov1987theorem,baskakov1994spectral,baskakov2002splitting}
Результатом решения этих уравнений
является
\begin{thm}[О локализации спектра матрицы смеж\-ностей
  почти-полного ор\-графа]\label{nk:thm:almost-all-ones}
      Пусть \( M < \frac{1}{16} N^2 \),
    где \( M \) --- число удалённых рёбер, а \( N \) --- число вершин в графе,
    тогда спектр матрицы \( A \) смежностей почти-полного орграфа можно представить в~виде
    объединения \( \sigma\left(A\right) = \sigma_1 \cup \sigma_2 \)
    непересекающихся
    одноэлементного множества \( \sigma_1=\{\lambda_1\} \)
    и~множества \( \sigma_2 \), удовлетворяющих условиям:
    \[ \sigma_1 \subset \left\{ \mu\in\mathbb{R}; \lvert \mu - N \rvert < 4\sqrt{M} \right\}, \]
    \[ \sigma_2 \subset \left\{ \mu\in\mathbb{C}; \lvert \mu \rvert < 4\sqrt{M} \right\}. \]
\end{thm}

Провед\"ем предварительные преобразования.

\begin{lem}
    Матрица единиц 
    \( \mathcal{J}_N =
    \begin{pmatrix}
        1 & \cdots & 1 \\
        \vdots & \ddots & \vdots \\ 
    1 & \cdots & 1 \end{pmatrix} \),
    подобна матрице
    \[
        \mathcal{A} = \begin{pmatrix}
            N & 0 & \cdots & 0 \\
            0 & 0 & \cdots & 0 \\
            \vdots & \vdots & \ddots & \vdots \\
            0 & 0 & \cdots & 0 \end{pmatrix}. \]
    Точнее, существует ортогональная матрица \( \mathcal{U} \),
    такая что
    \( \mathcal{J}_N = \mathcal{U}\mathcal{A} \mathcal{U}^{-1} \).
\end{lem}
\begin{proof}
  Единственному ненулевому собственному значению \( N \) соответствует
  одномерное собственное подпространство, порождённое собственным вектором
  \[
    h_N = \frac{1}{\sqrt{N}} \left(1, \ldots, 1\right)\in\mathbb{R}^N.
  \]
  Ядро (null-space) матрицы \( \mathcal{J}_N \) ортогонально вектору \( h_N \)
  и порождается базисом
  \( f_1 = {\left(1,-1,0,\ldots,0\right)}, \ldots,
     f_{N-1} = {\left(0,\ldots,0,1,-1\right)}. \)
  Применив ортогонализацию Грама-Шмидта, получим ортонормальную систему \( h_1, \ldots, h_N \):
    \[
        h_k = \frac{1}{\sqrt{k(k+1)}}
            \left(\smash{\underbrace{1,~\ldots,~1,}_{k\ \text{copies}}}~-k,~0,~\ldots,~0\right)
            \in \mathbb{R}^N, \quad k={1, \ldots, N-1} \]
    \[
        h_N = {\left(1,~\ldots,~1\right)} \in \mathbb{R}^N, \]

Отсюда заключаем, что матрица смежностей \( A = \mathcal{J}_N - B \)
рассматриваемого графа подобна матрице
    \( \mathcal{A} - \mathcal{B} \)
    где \( A \) --- блочная матрица:
    \[
        \mathcal{A} = \left(\begin{array}{c|c}
        N & \mathbf{0}_{\mathtt{Mat}_{1{\times}(N{-}1)}(\mathbb{C})}\\ \hline
            \mathbf{0}_{\mathtt{Mat}_{(N{-}1){\times}1}(\mathbb{C})} & \mathbf{0}_{\mathtt{Mat}_{(N{-}1){\times}(N{-}1)}(\mathbb{C})}
        \end{array}\right) \in \mathtt{Mat}_{N{\times}N}(\mathbb{R})
    \]
    а \( \mathcal{B} \) получается применение преобразования подобия к матрице
    \( B \):
    \(
        \mathcal{B} = U^{-1} B U \in \mathtt{Mat}_{N{\times}N}(\mathbb{R}).
    \)
Матрица подобия \( U \) имеет столбцами собственные векторы \( \mathcal{J}_N \):
    \[
        U = \operatorname{columns}(h_N, h_1, \ldots, h_{N-1}) =
        \begin{pmatrix}
            \vline & \vline &        & \vline \\
            h_N    & h_1    & \ldots & h_{N-1} \\
            \vline & \vline &        & \vline

        \end{pmatrix}.
    \]


Таким образом, исходная матрица \( \mathcal{A} \) подобна матрице
\( \mathcal{A} - \mathcal{B} \), где \( \mathcal{B} = \mathcal{U}^{-1} B \mathcal{U} \).
\end{proof}

Далее ортогональность матрицы \( U \) будет играть важную роль.

Матрицы из \( \mathtt{Mat}_N(\mathbb{C}) \) будем записывать в~блочном виде
\[ X \sim
    \begin{pmatrix}
    x_{11} & X_{12} \\
    X_{21} & X_{22}
    \end{pmatrix}, \]
    где \[ x_{11} \in \mathtt{Mat}_{1{\times}1}(\mathbb{C})\ \text{--- число}, \]
    \[ X_{12} \in \mathtt{Mat}_{1{\times}(N-1)}(\mathbb{C})\ \text{--- строка}, \]
    \[ X_{21} \in \mathtt{Mat}_{(N-1){\times}1}(\mathbb{C})\ \text{--- столбец}, \]
    \[ X_{22} \in \mathtt{Mat}_{(N-1){\times}(N-1)}(\mathbb{C}). \]
Её блоки имеют естественную матричную структуру с левым и правым умножением,
равно как составленные из них квадратные блочные матрицы имеют структуру
Банаховой алгебры, изоморфной \( \mathtt{Mat}_{N{\times}N}(\mathbb{C})\).
Кроме того, их можно естественным образом умножать на
на элементы пространства \( \mathbb{C}\times\mathbb{C}^{N-1} \),
изоморфного~\( \mathbb{C}^N \), если рассматривать его как
\( \mathtt{Mat}_{N{\times}1}(\mathbb{C})\):
\[
    \begin{pmatrix}
    x_{11} & X_{12} \\
    X_{21} & X_{22}
    \end{pmatrix}
    \begin{pmatrix} x_1 \\ x_2 \end{pmatrix}
  = \begin{pmatrix}
      x_{11} x_1 + X_{12} x_2 \\
      X_{21} x_1 + X_{22} x_2
      \end{pmatrix},\quad x \sim \begin{pmatrix} x_1 \\ x_2 \end{pmatrix}\in \mathbb{C}\times\mathbb{C}^{N-1}.
    \]
В~дальнейших выкладках изоморфные объекты понимаются взаимозаменяемыми и не различаются.

Следуя общей схеме метода подобных операторов~\cite{baskakov1986theorem,baskakov1987theorem,baskakov1994spectral,baskakov2002splitting},
будем искать более ``простую'' матрицу, подобную \( \mathcal{A} - \mathcal{B} \),
в~виде \( \mathcal{A} - J X \)
с~матрицей преобразования подобия \( E + \Gamma X \),
где \( E\in{\mathtt{Mat}_{N{\times}N}(\mathbb{C})} \)~--- единичная матрица,
\( J,\Gamma : \mathtt{Mat}_{N{\times}N}(\mathbb{C})\to\mathtt{Mat}_{N{\times}N}(\mathbb{C}) \)~--- линейные операторы,
действующие на алгебре \( \mathtt{Mat}_{N{\times}N}(\mathbb{C}) \), подбираемые
в~ходе решения,
      прич\"ем \( J \) --- проектор (\(J^2=J\)),
      ``упрощающий'' возмущение \( JX \),
      а \( \Gamma \)
      при всех \( X\in {\mathtt{Mat}_{N{\times}N}(\mathbb{C})} \) % \( X\sim \begin{pmatrix}x_{11} & X_{12} \\ X_{21} & X_{22}\end{pmatrix} \in {\mathtt{Mat}_{N{\times}N}(\mathbb{C})} \)
      удовлетворяет уравнению
          \( \mathcal{A}\Gamma X - (\Gamma X) \mathcal{A} = X - JX. \)

% Короче, будем решать в~Банаховой алгебре матриц порядка \( N \) уравнение
% \begin{equation}\label{nk:eq:similarity}
%     (\mathcal{A-B})(E+\Gamma X) = (E+\Gamma X)(\mathcal{A} - J X), \quad X\in\mathtt{Mat}_{N{\times}N}(\mathbb{C}).
%     \end{equation}
% Оператор \( J \) обычно выбирают проектором (\(J^2=J\)).
% \( \Gamma \) определяют поточечно, как решение уравнения
% \( \mathcal{A}\Gamma X - (\Gamma X) \mathcal{A} = X - J X, \quad X\in\mathtt{Mat}_{N{\times}N}(\mathbb{C}) \),
% где \( \mathcal{A}\Gamma X - (\Gamma X) \mathcal{A} = \mathtt{ad}_{\mathcal{A}} \Gamma X \),
% \(  \mathtt{ad}_{\mathcal{A}}: \mathtt{Mat}_{N{\times}N}(\mathbb{C})\to\mathtt{Mat}_{N{\times}N}(\mathbb{C}) \)
% --- оператор коммутирования с \( \mathcal{A} \).
% Ясно, \( \mathcal{A}-JX \) имеет тем более простую структуру,
% чем шире ядро проектора \( J \).
% Уравнение для \( \Gamma \) в~свою очередь не позволяет сузить ядро слишком сильно.

\begin{lem}
    Операторы \( J \) и \( \Gamma \)
    следует задать формулами
    \[
        J X = \begin{pmatrix} x_{11} & 0 \\ 0 & X_{22} \end{pmatrix}, \]
    \[
        \Gamma X = \frac{1}{N} \begin{pmatrix} 0 & X_{12} \\ -X_{21} & 0 \end{pmatrix}, \]
        для \( X\sim \begin{pmatrix}x_{11} & X_{12} \\ X_{21} & X_{22}\end{pmatrix} \in \mathtt{Mat}_{N{\times}N}(\mathbb{C}) \).

\end{lem}
\begin{crl}
    Спектр блочно-диагональной матрицы
    \[ \mathcal{A} - JX = \begin{pmatrix} N - x_{11} & 0 \\ 0 & X_{22} \end{pmatrix} \]
    есть объединение спектров е\"е диагональных блоков:
    \[
        \sigma(\mathcal{A} - J X) = \{ N - x_{11} \} \cup \sigma(X_{22}). \]
\end{crl}
\begin{proof}
Пусть \( \Gamma \) действует по формуле
\[ \Gamma X = \begin{pmatrix} \Gamma_{11}(X) & \Gamma_{12}(X) \\
                              \Gamma_{21}(X) & \Gamma_{22}(X)
                            \end{pmatrix}, \]
тогда уравнение для \( \Gamma X \) приобретает вид
    \[
        \mathcal{A} \Gamma X - (\Gamma X)\mathcal{A} =
        N
        \begin{pmatrix}
          0 & \Gamma_{12}(X) \\
          -\Gamma_{21}(X) & 0
        \end{pmatrix} = X - JX,
     \]
     и, так как \( J\Gamma X = 0 \), получаем соотношение
\[
    X - J X =
    N \begin{pmatrix} 0 & \Gamma_{12}(X) \\
        - \Gamma_{21}(X) & 0
        \end{pmatrix}. \]

Значит, \( J \) может обнулить в
    \( X \sim
    \begin{pmatrix}
    x_{11} & X_{12} \\
    X_{21} & X_{22}
    \end{pmatrix} \in \mathtt{Mat}_{N{\times}N}(\mathbb{C}) \)
    вс\"е, кроме двух диагональных блоков \( x_{11} \) и \( X_{22} \),
    поэтому положим
\[
    J X = \begin{pmatrix} x_{11} & 0 \\ 0 & X_{22} \end{pmatrix}, \]
\[
    \Gamma X = \frac{1}{N} \begin{pmatrix} 0 & X_{12} \\ -X_{21} &
      0 \end{pmatrix},\ X\in\mathtt{Mat}_{N{\times}N}(\mathbb{K}). 
\]
\end{proof}

Теперь выпишем уравнение подобия матриц \( \mathcal{A} - \mathcal{B} \)
и \( \mathcal{A} - J X \):
\begin{equation}\label{nk:eq:similarity}
    (\mathcal{A-B})(E+\Gamma X) = (E+\Gamma X)(\mathcal{A} - J X), \quad X\in\mathtt{Mat}_{N{\times}N}(\mathbb{C}).
\end{equation}
\begin{lem}
    Уравнение~\eqref{nk:eq:similarity} эквивалентно уравнению
    \begin{equation}\label{nk:eq:fixptn}
        X = \mathcal{B} \Gamma X + \mathcal{B} - (\Gamma X)(J(\mathcal{B} (E + \Gamma X))), \quad X\in\mathtt{Mat}_{N{\times}N}(\mathbb{C}).
    \end{equation}
\end{lem}
\begin{proof}
Раскрывая скобки, уравнение~\eqref{nk:eq:similarity} можно преобразовать к виду
\begin{equation}\label{nk:eq:fixptn-ini}
    X = \mathcal{B} \Gamma X + \mathcal{B} - (\Gamma X) J X.
\end{equation}
Пусть для \( X \) выполнено~\eqref{nk:eq:fixptn-ini}.
Тогда, учитывая равенство \( J\left((\Gamma X)JX\right) = 0, \)
получим равенство
    \begin{equation}\label{nk:eq:jx}
        J X = J\mathcal{B} + J\left(\mathcal{B}\Gamma X\right) = J(\mathcal{B} (E + \Gamma X)).
    \end{equation}
Подставляя это выражение обратно в~\eqref{nk:eq:fixptn-ini},
    получим~\eqref{nk:eq:fixptn}.
Аналогично, применяя к обеим частям равенства~\eqref{nk:eq:fixptn} оператор \( J \)
    и учитывая, что \( J\left( (\Gamma X)J(\mathcal{B} (E + \Gamma X)) \right) = 0 \),
    получим~\eqref{nk:eq:fixptn-ini}
\end{proof}

Выражение в правой части уравнения~\eqref{nk:eq:fixptn} обозначим как
\[
    \Phi(X) = \mathcal{B} \Gamma X + \mathcal{B} - (\Gamma X)(J(\mathcal{B} (E + \Gamma X))).\]
Теперь покажем, что, при определ\"енных условиях,
возникшее нелинейное отображение
\[ \Phi:\mathtt{Mat}_{N{\times}N}(\mathbb{C})\to \mathtt{Mat}_{N{\times}N}(\mathbb{C}) \]
имеет инвариантным множеством
некоторый шар \( \Omega \subset \mathtt{Mat}_{N{\times}N}(\mathbb{C}) \) с~центром в~нуле
(т.е.~\( \Phi(\Omega)\subset\Omega \)),
на котором оно является сжимающим.

Пусть в~\( \mathtt{Mat}_{N{\times}N}(\mathbb{C}) \)
выбрана какая-нибудь суб\-мульти\-пликативная норма \( \|\cdot\| \)
(т.е.~норма, удовле\-творя\-ющая неравенству
 \( \| \mathcal{A}_1\mathcal{A}_2 \| \leq \|\mathcal{A}_1\|\|\mathcal{A}_2\|, \)
 \( \mathcal{A}_1, \mathcal{A}_2 \in \mathtt{Mat}_{N{\times}N}(\mathbb{C}) \)).
Нам нужно найти такой радиус \( r \geq 0 \),
что при \[ \|X\|,\|Y\| \leq r \]
выполнялись бы неравенства
\[ \|\Phi(X) - \Phi(Y)\| < q\|X-Y\|,\ q\in(0..1), \]
\[ \|\Phi(X)\| \leq r. \]
Обозначим
\( \beta = \|\mathcal{B}\| \), \( \gamma = \sup_{\|X\|=1} \|\Gamma X\| \).

\begin{lem}
    Пусть \( \gamma\beta < \frac14\),
    тогда шар
    \[
        \Omega = \left\{ X\in \mathtt{Mat}_{N{\times}N}(\mathbb{C}); \|X\| \leq r_0 \right\}, \]
    \[  0 < r_0 = \frac{1 - 2\gamma\beta - \sqrt{1-4\gamma\beta}}{2\gamma^2\beta} < 4\beta, \]
    удовле\-творяет условию \( \Phi(\Omega)\subset\Omega \).
\end{lem}
\begin{proof}
Очевидно неравенство
    \[ \| \Phi(X) \| \leq
     \beta \gamma^2 \|X\|^2 + 2\beta\gamma\|X\| + \beta. \]
Значит, если \( r \) удовле\-творяет неравенству
    \begin{equation}\label{nk:ineq:invariance-radius}
        \beta \gamma^2 r^2 + (2\beta\gamma - 1)r + \beta \leq 0,
    \end{equation}
    то \( \|\Phi(X)\| \leq r \) при всех \( \|X\| \leq r \).
Если \( \gamma\beta \leq \frac14 \),
    то ``дискриминант'' \( \Delta = 1-4\gamma\beta \)
    соответствующего уравнения положителен и~его корни вещественны.
Из знаков коэффициентов возникшего многочлена видно, что оба корня положительны.
Следовательно, наименьший положительный~\( r \),
    удовле\-творя\-ющий неравенству~\eqref{nk:ineq:invariance-radius}
    есть наименьший корень
    соответствующего уравнения:
    \[ r_0 = \frac{1 - 2\gamma\beta - \sqrt{1-4\gamma\beta}}{2\gamma^2\beta}. \]
Учитывая \( \gamma\beta<\frac14 \), имеем \( r_0 < 4\beta \).
\end{proof}

Аналогичным образом устанавливается
\begin{lem}
    Пусть \(\gamma\beta<\frac14\),
    тогда \( \Phi \)~--- сжимающее отображение:
    \[ \| \Phi(X) - \Phi(Y) \| \leq q \|X - Y\|, \quad X,Y\in\Omega \]
    \[ q = (1+2\gamma r_0) \gamma\beta \leq (1+8\gamma\beta)\gamma\beta \leq \frac34. \]
\end{lem}
\begin{proof}
    \begin{align*} \| \Phi(X) - \Phi(Y) \| = \| \mathcal{B}\Gamma (X-Y) + (\Gamma X)(\mathcal{B}\Gamma X + \mathcal{B})
     - (\Gamma Y)(\mathcal{B} \Gamma Y + \mathcal{B}) \| \leq \\
        \leq
     \beta\gamma\|X-Y\| +
     \beta \gamma^2 \|X-Y\| \|X+Y\| \leq \\
        \leq
     \beta\gamma\|X-Y\| +
      2 r_0 \beta \gamma^2 \|X-Y\|.
    \end{align*}
Здесь использовано равенство
\[ (\Gamma X) J(\mathcal{B}\Gamma X) - (\Gamma Y) J(\mathcal{B}\Gamma Y) =
  \]
 \[= \frac12\left[
        \Gamma(X-Y) J(\mathcal{B}\Gamma(X+Y))
    +   \Gamma(X+Y) J(\mathcal{B}\Gamma(X-Y))
  \right].
\]
\end{proof}

Отсюда и~из теоремы Банаха о~неподвижной точке следует:
\begin{lem}
В~шаре \[ \Omega = \left\{ X\in\mathtt{Mat}_{N{\times}N}(\mathbb{C}); \quad \|X\| \leq r_0 \right\} \]
    существует и~при том единственное решение \( X^o \) уравнения~\eqref{nk:eq:fixptn},
    являющееся пределом последовательности \( \{ \Phi^k(0); k\in\mathbb{N} \} \),
    где \( \Phi^k = \Phi\circ\Phi^{k-1} \)~--- композиция.
\end{lem}

\begin{crl}
Матрица \( \mathcal{A} - \mathcal{B} \) подобна блочно-диагональной матрице \( \mathcal{A} - J X^o \):
\[ \mathcal{A} - \mathcal{B} \sim
\begin{pmatrix}
N - x_{11}^o & 0 \\
0 & -X_{22}^o
\end{pmatrix}, \]
при этом выполняются условия:
\[ \sigma\left(\mathcal{A} - \mathcal{B}\right) = \left\{N-x_{11}^o\right\}\cup \sigma\left(-X_{22}^o\right), \]
    \[ x_{11}^o\in\mathbb{R}, \lvert x_{11}^o \rvert \leq r_0 \leq 4\beta, \]
\[ \sigma\left(-X_{22}^o\right) \subset \{ \mu\in\mathbb{C}; \lvert x \rvert \leq r_0 \leq 4\beta \}. \]
\end{crl}
\begin{proof}
    Матрица \( \mathcal{A} - \mathcal{B} \) подобна блочно-диагональной \( \mathcal{A} - J X^o \),
    поэтому их спектры совпадают.
    Спектр матрицы \( \mathcal{A} - J X^o \) есть объединение спектров е\"е диагональных блоков.
    В~виду субмультипликативности нормы имеют место неравенства
    \[ \mathtt{spr}(X^o) = \max_{\lambda\in\sigma(X^o)}\lvert\lambda\rvert \leq \|X^o\| \leq r_0. \]
    Кроме того, собственное значение \( x_{11}^o \) является вещественным, как предел сходящейся вещественной последовательности.
\end{proof}

Верн\"емся, наконец, к~непосредственному доказательству основной теоремы:
\begin{proof}[Доказательство Теоремы~\ref{nk:thm:almost-all-ones}]
    Рассмотрим в~пространстве \( \mathtt{Mat}_{N}(\mathbb{C}) \)
    норму Фробениуса \( {\left\|\cdot\right\|}_{F} \),
    определ\"енную формулой
    \[ {\left\|X\right\|}_{F} = \sqrt{\sum_{ij} \lvert x_{ij}\rvert^2}, \]
    \( X = (x_{ij})\in\mathtt{Mat}_{N{\times}N}(\mathbb{C}) \).
    Она суб\-мульти\-пли\-кативна.
    Заметим, что матрица \( \mathcal{U} \),
    приводящая \( \mathcal{J}_N \) к~диагональному виду
    является ортогональной,
    поэтому умножение на \( \mathcal{U} \) или \(\mathcal{U}^{-1}\)
    есть изометрия относительно \( \|\cdot\|_F \) и поэтому
    \[ \|\mathcal{B}\|_F=\|B\|_F. \]
    Наконец,
    \( B \) состоит из \( M \) единиц, поэтому
    \[
        \beta = {\left\|B\right\|}_{F} =
        {\left\|B\right\|}_{F} = \sqrt{M}.
        \]
    Заметим также очевидное равенство
    \[
        \gamma = \frac1N
                \sup_{{\left\|X\right\|}_{F}=1}{\left\|\begin{pmatrix}0 & X_{12} \\ -X_{21} & 0\end{pmatrix}\right\|}_{F}
                = \frac1N. \]
    
    Если
     \( \sqrt{M} < \frac{N}{4} \),
     то выполняются условия леммы,
     прич\"ем \( r_0 < 4\sqrt{M} \).
    Это значит, что
     \( \sigma(A) = \sigma_1 \cup \sigma_2 \),
     где \( \sigma_1 = \{ \lambda_1 \}\subset\mathbb{R}, \lvert \lambda_1 - N \rvert < 4\sqrt{M} \),
     \( \sigma_2 \subset \{ \mu\in\mathbb{C}; \lvert\mu\rvert < 4\sqrt{M} \} \),
     \( \sigma_1 \cap \sigma_2 = \emptyset \).

Дополнительно можно сказать, что 
     главный собственный вектор \( A \) имеет вид
     \[
       \hat{h}_N = U(E+\Gamma X^o) e_1 =
       h_N - \frac1N (X_{21,(1)}^o h_1 + \cdots + X_{21, (N{-}1)}^o h_{N{-}1}),
     \]
     где \( X_{21,(i)}^o,\ i=\overline{1,N{-}1} \) --- координаты вектора
     \( X_{21}^o \).
     Более того \( \hat{h}_N\in\mathbb{R}^{N} \),
     удовлетворяет неравенствам
     \[
       \|\hat{h}_N - h_N\|_2 \leq 4\frac{\sqrt{M}}{N},
     \]
     Здесь
     \[ X^o =
       \begin{pmatrix}
         x_{11}^o & X_{12}^o \\
         X_{21}^o & X_{22}^o
       \end{pmatrix}
       = \left( \Phi^k(0);\ k\in\mathbb{N} \right), \]
     \[
       \Phi(X) = B\Gamma X - (\Gamma X)J(B + B\Gamma X) + B.
     \]
     
     Теорема доказана.
   \end{proof}

\newpage
% JPCS
\section{``A-tiled'' матрица}

Пусть теперь \( A\in\mathtt{Mat}_{M{\times}M}(\mathbb{K}) \).
Рассмотрим следующую (``A-tiled'') блочную матрицу:
    \[
        \mathbb{A} =
        \begin{pmatrix}
            A & \cdots & A \\
            \vdots & \ddots & \vdots \\
            A & \cdots & A
        \end{pmatrix}
        \in\mathtt{Mat}_{{MN}{\times}{MN}}(\mathbb{K})
    \]
    и возмущённую матрицу
    \[
        \mathbb{A} - \mathbb{B},\ \mathbb{B}\in\mathtt{Mat}_{{MN}{\times}{MN}}(\mathbb{K}).
    \]

\begin{lem}
    Пусть \( A \) обратима и самосопряжена.
    Тогда она имеет \( M \) ортонормальных собственных векторов \( h_1, \ldots, h_M \)
    (\(\left\|h_i\right\|_2 = 1,\ i{=}\overline{1,M}\))
    которым соответствует (не обязательно различные) собственные значения
    \( \lambda_1, \ldots, \lambda_M \neq 0\).

    Спектр \( \mathbb{A} \) есть
    \[
        \sigma(\mathbb{A}) = \{0\}\cup N\sigma(A) = \{0\} \cup \{N\lambda;\ \lambda\in\sigma(A) \}.
    \]
    Ненулевые собственные значения матрицы \( \mathbb{A} \)
    соответствуют следующим блочным собственным векторам:
    \[
        f_j = \frac{1}{\sqrt{N}} (h_j, \ldots, h_j)\in \mathbb{K}^{MN},\ j=\overline{1,M}.
    \]
    Ядро \( \mathbb{A} \)
    также обладает ортобазисом:
    \[
        f_{j,k} = \frac{1}{\sqrt{k(k+1)}}
        (
        \underbrace{e_j, \ldots, e_j}_{k\ \text{copies}},
        -ke_j,
        0, \ldots, 0
        ) \in\mathtt{Mat}_{{MN}{\times}{MN}}(\mathbb{K})
    \]

    Матрица \( \mathbb{A} \) таким образом подобна блочной матрице:
    \[
        \mathcal{A} =
        \left(\begin{array}{c|c}
            \operatorname{diag}(N\lambda_1,\ldots,N\lambda_M) & \mathbf{0} \\ \hline
            \mathbf{0} & \mathbf{0}
        \end{array}\right)\in\mathtt{Mat}_{{MN}{\times}{MN}}(\mathbb{K});
    \]
    Матрица подобия имеет вид
    \[
        U = \operatorname{columns}
        \left(f_1, \ldots, f_M, f_{1,1}, \ldots, f_{1,N{-1}}, \ldots, f_{M,N{-}1}\right).
    \]
\end{lem}

В этой секции мы рассматриваем матрицы размера
    \( {MN}{\times}{MN} \)
    в блочной форме
    \[
    X =
        \left(\begin{array}{c|c}
              X_{11} & X_{12} \\ \hline
              X_{21} & X_{22} \\
        \end{array}\right),
    \]
где
\[ X_{11} \in \mathtt{Mat}_{M{\times}M}(\mathbb{K}).\]
\[ X_{12} \in \mathtt{Mat}_{M(N-1){\times}M}(\mathbb{K}).\]
\[ X_{21} \in \mathtt{Mat}_{M{\times}M(N-1)}(\mathbb{K}).\]
\[ X_{22} \in \mathtt{Mat}_{M(N-1){\times}M(N-1)}(\mathbb{K}).\]

Повторим процедуру.
Естественным выбором \( J \) является
\[
        J X =
        \left(\begin{array}{c|c}
            \begin{matrix}
                x_{11} &  & 0 \\
                 & \ddots &  \\
                0 &  & x_{MM}
            \end{matrix} &
            \begin{matrix}
                0 \\
            \end{matrix} \\ \hline
            \begin{matrix}
                0 &
            \end{matrix} &
            X_{M+1,M+1}
        \end{array}\right).
\]

\begin{lem}
Пусть \( A \) --- простого спектра,
    i.e.\ он имеет \( M \) различных собственных значений:
    \( \lambda_i\neq\lambda_j \) for all \( 1\leq i{\neq}j \leq M \).

    Тогда тройка \( (\mathtt{Mat}_{{MN}{\times}{MN}}(\mathbb{K}), J, \Gamma) \)
    будет допустимой, если мы положим
    \[
        \Gamma X = 
        \frac1n \left(\begin{array}{c|c}
                        \Gamma_{11}(X) & \Gamma_{12}(X) \\ \hline
                        \Gamma_{21}(X) & \mathbf{0}
                      \end{array}\right),
                                      \]
    \[
        \Gamma_{11}(X) =
              \begin{pmatrix}
                0               & \gamma_{12}x_{12} & \cdots & \gamma_{1M}x_{1M} \\
                \gamma_{21}x_{21}  & 0              & \cdots & \gamma_{2M}x_{2M} \\
                \vdots          & \vdots         & \ddots & \vdots & \ \\
                \gamma_{M1}x_{M1}  & \gamma_{M2}x_{M2} & \cdots & 0
              \end{pmatrix}
    \]
    \[
        \Gamma_{12}(X) =
            \begin{pmatrix}
                \gamma_{1}x_{1,M+1} & \cdots & \gamma_{1}x_{1,M(N-1)} \\
                \gamma_{2}x_{2,M+1} & \cdots & \gamma_{2}x_{2,M(N-1)} \\
                \vdots                & \ddots &  \vdots                \\
                \gamma_{M}x_{M,M+1} & \cdots & \gamma_{M}x_{M,M(N-1)}
            \end{pmatrix}
    \]
    \[
        \Gamma_{21}(X) =
            - \begin{pmatrix}
                \gamma_{1}x_{M{+}1,1} &
                \cdots &
                \gamma_{M}x_{M{+}1,M}
                \\
                \vdots & \ddots & \vdots
                \\
                \gamma_{1}x_{M(N-1),1} &
                \cdots &
                \gamma_{M}x_{M(N-1),M}
            \end{pmatrix}
    \]
    \[
        \Gamma_{22}(X) = \mathbf{0},
    \]
    \[
        \gamma_{ij} = \left\{
            \begin{aligned}
                & \frac{1}{\lambda_i - \lambda_j},\ 1\leq i{\neq}j \leq M{+}1,\\
                & 0,\ i=j
            \end{aligned}
            \right.
    \]
    \[
        \gamma_{i} =
                \frac{1}{ \lambda_i},\ 1\leq i \leq M,
    \]
    где мы использовали соглашение
    \[
        \lambda_{M{+}1} = 0.
    \]

    Вычислим операторную норму \( \Gamma \):
    \[
        \|\Gamma\|_{\mathrm{op}} =
        \frac1N
        \frac{1}{\min\limits_{1\leq i{\neq}j \leq M{+}1}|\lambda_i - \lambda_j|} =
        \]
    \[
        = \frac1N
         \max\left\{
         \frac{1}{
             \min\limits_{1\leq i{\neq}j \leq M }{|\lambda_i - \lambda_j|}},
         \frac{1}{
             \min\limits_{1\leq j \leq M}{|\lambda_j|}}
         \right\}
        \]
\end{lem}

\begin{thm}[О локализации спектра возмущённой ``A-tiled'' матрицы]\label{nk:thm:tiled}
  Пусть \( A \) --- простого спектра,
    \[
        \mathbb{A} =
        \begin{pmatrix}
            A & \cdots & A \\
            \vdots & \ddots & \vdots \\
            A & \cdots & A
        \end{pmatrix}
        \in\mathtt{Mat}_{{MN}{\times}{MN}}(\mathbb{K}),
    \]
и имеет место следующее неравенство:
\[
    \left\| \mathbb{B} \right\|_{\mathrm{op}}
        \leq 
        \frac{N}{4}
         \min\left\{
             \min\limits_{1\leq i{\neq}j \leq M }{|\lambda_i - \lambda_j|},
             \min\limits_{1\leq j \leq M}{|\lambda_j|}
         \right\}.
 \]

Тогда спектр пертурбированной матрицы \( \mathbb{A} - \mathbb{B} \) представим в виде
\[
    \sigma\left(\mathbb{A}\right) =
        \left\{
            N\lambda_1 - x_{11}^o, \ldots, N\lambda_M - x_{MM}^o
        \right\}
    \cup \sigma_{M{+}1}.
\]

Собственные векторы
    \( \hat{f}_j,\ \hat{f}_{j,k},\ j{=}\overline{1,M},\ k{=}\overline{1,N{-1}} \)
    матрицы \( \mathbb{A}{-}\mathbb{B} \),
    значения \( x_{jj}^o,\ j{=}\overline{1,M} \)
    и множество \( \sigma_{M{+}1} \) удовлетворяют следующим ограничениям:
\[
    \lvert x_{jj}^o\rvert
    \leq 4\|B\|, \]
\[  \ \max_{\lambda\in\sigma_{M{+}1}} \lvert\lambda\rvert
    \leq 4\|B\|,
\]
\[
  d = 
    \frac4N \|B\|
         \max\left\{
         \frac{1}{
             \min\limits_{1\leq l{\neq}p \leq M }{|\lambda_l - \lambda_p|}},
         \frac{1}{
             \min\limits_{1\leq l \leq M}{|\lambda_l|}}
         \right\}
    \]
\[
    \left\| \hat{f}_j - f_j \right\|_2
    \leq d
    \]
\[
    \ \left\| \hat{f}_{j,k} - f_{j,k}\right\|_2
    \leq d
\]
Для всех \( j{=}\overline{1,M}, k{=}\overline{1,N-1} \).

\end{thm}

\newpage
\section{Произведения Кронекера}

Рассмотрим теперь Кронекеровы произведения
\[
    A\otimes B =
    \begin{pmatrix}
        a_{11} B & \cdots & a_{1N} B \\
        \vdots   & \ddots & \vdots \\
        a_{N1} B & \cdots & a_{NN} B
    \end{pmatrix}
    \in \mathtt{Mat}_{{MN}{\times}{MN}}(\mathbb{K})
\]
квадратных матриц
\( A={(a_{ij})}\in\mathtt{Mat}_{N{\times}N}(\mathbb{K}),
 \ B={(b_{ij})}\in\mathtt{Mat}_{M{\times}M}(\mathbb{K}). \)
Будем исследовать спектральные свойства таких матриц под действием небольших по
норме возмущений:
\begin{equation}\label{-kronperturb}
    A\otimes B - F.
\end{equation}

% TODO:
% We will also address special
%     perturbations of the following form (cf.~\cite{XIANG2005210}):
% \[
%     (A-\Delta A)\otimes (B - \Delta B).
% \]

Обратим внимание на некоторые полезные свойства Кронекерова произведения~\cite{bellman-matrices-kron}.
\begin{itemize}
\item Ассоциативность:
    \[ A\otimes (B\otimes C) = (A\otimes B)\otimes C. \]
\item Дистрибутивность относительно сложения:
    \[ (A+B)\otimes(C+D) = A\otimes C + A\otimes D + B\otimes C + B\otimes D. \]
\item Равенство
    \[ (AB)\otimes(CD) = (A\otimes C)(B\otimes D) \]
    имеет место
    всегда, когда произведения \( AB \) и \( CD \) ``имеют смысл'' и существуют.
  \item След (trace) матрицы \( A\otimes B \) есть
    \[ \operatorname{tr}(A\otimes B) = \operatorname{tr}A\operatorname{tr}B. \]
\item Если \( A \) и \( B \) --- симметричные,
      то и произведение \( A\otimes B \) симметрично.
\end{itemize}

Заметим, что ``tiled'' matrix из последнего примера
можно представить как Кронекерово произведение:
\[
    \mathbb{A} =
    \begin{pmatrix}
    A & \cdots & A\\
    \vdots & \ddots & \vdots \\
    A & \cdots & A\end{pmatrix} =
        \mathcal{J}_N\otimes A.
    \]

\begin{lem}
Пусть \( A \) и \( B \) --- простой структуры,
    i.e. \( A \) имеет \( N \) независимых собственных векторов
    \( f_1, \ldots, f_N \),
    которым соответствуют собственные числа \( \mu_1, \ldots, \mu_N \)
    и \( B \) имеет собственные векторы \( h_1, \ldots, h_M \)
    с собственными значениями \( \lambda_1, \ldots, \lambda_M \).

Тогда произведение \( A\otimes B \) --- также простой структуры;
    оно имеет \( MN \) независимых Eigenvektoren \( f_i\otimes h_j,\ i{=}\overline{1,N}, j{=}\overline{1,M} \)
    которым соответствуют Eigenwerte \( \mu_i \lambda_j \).
\end{lem}


Сгруппируем одинаковые собственные значения: пусть среди произведений \( \mu_i
\lambda_j \) лишь \( s \) различных чисел: \( \nu_1, \ldots, \nu_s \).
Каждому собственному значению \( \nu_k \) (\( k{=}\overline{1,s} \)) соответствует
собственное подпространство (eigenspace, Eigenraum)
\[ E_k = \operatorname{span}(f_i\otimes h_j;\ \mu_i\lambda_j = v_k,\ i{=}\overline{1,N},\ j{=}\overline{1,M}) \subset \mathbb{K}^{MN}. \]
Эти подпространства составляют разложение \( \mathbb{K}^{MN} \) в прямую сумму:
    \[ \mathbb{K}^{MN} = E_1 \oplus \cdots \oplus E_s. \]
Всякий вектор \( x\in\mathbb{K}^{MN} \) можно единственным образом представить в виде
    \begin{equation}\label{-decomposition-x}
        x = x_1 + \cdots + x_s,\ x_k\in E_k,\ k=\overline{1,s}.
    \end{equation}
Этой direct-sum decomposition пространства \( \mathbb{K}^{MN} \)
    соответствует декомпозиция единичной матрицы \( E\in \mathtt{Mat}_{MN{\times}MN}(\mathbb{K}) \)
    (соответствующей тождественному отображению)
    в прямую сумму идемпотентных матриц --- \emph{спектральных проекторов}:
    \[
        E = P_1 + \cdots + P_s.
    \]
Спектральная проекция \( \mathcal{P}_k \) (\(k{=}\overline{1,s}\)) определяется формулой
    \[
        \mathcal{P}_k x = x_k \in E_k\subset \mathbb{K}^{MN}
    \]
    в соответствие с разложением~\eqref{-decomposition-x} \( x \).

Для всякой матрицы \( X\in \mathtt{Mat}_{MN{\times}MN}(\mathbb{K}) \)
верно следующее неравенство:
    \[
        X = \sum_{i,j=1}^s P_i X P_j.
    \]

Матрица \( A\otimes B \) может быть разложена
    \[
        \mathcal{A} = \sum v_j P_j.
    \]

Наконец, мы готовы получить оценки

% \begin{center}
% \textbf{Лемма.}
% {\it
Как и прежде, вид \( J \) ``напрашивается'' сам:
    \[
        JX = \sum_{j=1}^s P_j X P_j.
    \]
Система уравнений
    \[\left\{\begin{aligned}
        & \mathcal{A}\Gamma X - (\Gamma X) \mathcal{A} = X - JX, \\
        & J\Gamma X = 0,\ X\in \mathtt{Mat}_{MN{\times}MN}(\mathbb{K})
    \end{aligned}\right.\]
имеет единственное решение:
    \[
        \Gamma X = \sum_{1\leq i{\neq}j \leq s} \frac{1}{\nu_i-\nu_j} P_i X P_j.
    \]
Норма \( \Gamma \) считается:
    \[
        \|\Gamma\|_{\mathrm{op}} = \gamma = \frac{1}{\min_{1\leq i{\neq}j\leq s}\lvert\nu_i - \nu_j\rvert}
    \]
% \/}
% \end{center}

Непосредственным образом выводим теорему:

\begin{thm}[О представлении спектра и локализации собственных пространств
  возмущений Кронекеровых произведений]\label{nk:thm:kron}
      Рассмотрим возмущённую матрицу~\eqref{-kronperturb}:
        \[
            A{\otimes}B - F.
        \]
    Пусть \( A\in\mathtt{Mat}_{N{\times}N}(\mathbb{K}) \) and \( B\in\mathtt{Mat}_{M{\times}M}(\mathbb{K}) \)
    --- диагонализуемые матрицы.
    Пусть \( f_1, \ldots, f_N \) --- eigen-векторы \( A \),
        которые соответствуют eigen-числам \( \mu_1, \ldots, \mu_N \),
        и пусть \( h_1, \ldots, h_M \) --- собственные векторы \( B \)
        с собственными числами \( \lambda_1, \ldots, \lambda_M \).
    Спектр их произведения Кронекера --- \( A{\otimes}B \) ---
        состоит из всевозможных попарных произведений \( \mu_i \lambda_j \)
        с соответствующими собственными векторами \( f_i\otimes h_j \).
    Сгруппируем эти \( MN \) произведений в \( s \) попарно различных
    собственных чисел:
        \( \nu_1, \ldots \nu_s \).

    Пусть
    \[
        \|F\| \leq \frac14 \gamma^{-1} = \frac14 \min_{1\leq i{\neq}j\leq s}\lvert\nu_i - \nu_j\rvert.
    \]

    Тогда матрица \( A{\otimes}B - F \) подобна матрице
    \[ \sum_{k=1}^s \nu_k P_k - JX^o = \sum_{k=1}^s (\nu_k P_k - P_k X^o P_k) \]
    для некоторого \( X^o \in \mathtt{Mat}_{MN{\times}MN}(\mathbb{K}) \),
    \( \|X^o - F\|\leq 3\|F\| \).

    Все собственные значения \( A{\otimes}B - F \) содержатся в шарах
    \[
        \Omega_k = \left\{
            \lambda\in\mathbb{C};
            \ \lvert\lambda - \nu_k\rvert \leq 4\|F\|
            \right\},
        \ k{=}\overline{1,s}.
    \]
    В каждом из шаров есть хотя бы одно собственное число.

    Пусть \( \nu_k=\mu_{i_k}\lambda_{j_k} \) --- собственное число \( A{\otimes}B \) кратности \( 1 \),
        id est ему соответствует единственный собственный вектор \( v_k = f_{i_k}{\otimes}h_{j_k} \).
    Эквивалентно: eigenvalue \( \mu_{i_k} \)
        матрицы \( A \) и eigenvalue \( \lambda_{j_k} \) матрицы \( B \)
        оба имеют кратность \( 1 \).
    Тогда \( A{\otimes}B - F \) имеет собственно значение в шаре \( \Omega_k \)
        которому соответствует собственный вектор \( \hat{v}_k \),
        удовлетворяющий неравенству
    \[
        \|\hat{v}_k - v_k\| \leq 4\gamma \|F\|.
    \]
    Если \( \nu_k \) хорошо отделено от прочего спектра \( A{\otimes}B \):
    \[
        \min_{l\neq k}
        \lvert
        \nu_k - \nu_l
        \rvert
        \geq 4\|F\|,\ \text{(спектральная отделимость)},
    \]
    то \( \nu_k \) --- единственное собственное значение \( A{\otimes}B - F \)
    в этом шаре.
\end{thm}

Например, в уже упомянутом случае ``tiled'' матрицы
\[
    \mathcal{J}_N{\otimes}B =
    \begin{pmatrix}
        B & \cdots & B \\
        \vdots & \ddots & \vdots \\
        B & \cdots & B
    \end{pmatrix}
\]
мы бы имели:
    \( \nu_1=N \),
    \( \nu_2=0 \).
Обозначим \( \lambda_1,\ldots,\lambda_M \)
    собственные значения \( B \).
Spectrum матрицы \( \mathcal{J}_N{\otimes}B \) есть
    \[
        \sigma(\mathcal{J}_N{\otimes}B) = \left\{ \mu_i\lambda_j;\ i{=}\overline{1,2},\ j{=}\overline{1,M}\right\} = \{0\}\cup N\sigma(B).
    \]
Все эти значения, за вычетом \( 0 \),
    имеют кратность \( 1 \)
    и хорошо отделены друг от друга при достаточных \( N \).
Получаем \( \gamma=\frac1N \).
Теорема предыдущего параграфа непосредственно следует из этого и последней теоремы.

Заметим, что результаты могут быть уточнены с помощью теоремы о расщеплении~\cite{baskakov1987theorem}
позволяющий отследить каждую отделённую часть спектра более ``независимо''
и получить более аккуратные оценки.
Точность метода повышается с увеличением спектральной отделимости.

\section{Вывод}

Локализованы собственные значения и векторы пертурбаций Кронекеровых
произведений диагонализуемых квадратных матриц. Для исследования был применён
метод подобных операторов.
Это оставляет возможность в будущем уточнить оценки, а также построить алгоритмы
для численного получения (evaluation of) собственных векторов и чисел.
Результаты также могут быть расширены на бесконечномерный случай.

\newpage
\appendix
\section{Определения}

\textbf{Функция} --- тройка \( (X, Y, f) \), состоящия из domain set \( X \),
the target set \( Y \), и правила \( f \), ставящего каждому \( x\in X\)
в соответствие некоторый \( y=f(x)\in Y \).

\textbf{Векторное пространство над полем \(\mathbb{K}\)} (в.п.) --- кортеж \( (X, 0, +, \cdot) \)
из множества \( X \), образующего Абелеву группу, относительно сложения \(
+:X\to X\), с нулём \( 0\in X \)
и скалированием (умножением на скаляр)
\(\cdot: \mathbb{K}\times X \to X \), обладающими естественными свойствами.

Для простоты мы часто будем обозначать пространство со структурой как просто \( X \).

Examples: координатные пространства \( \mathbb{K}^n,\ n\in\mathbb{N} \) с поэлементными
скалированием и сложением, полиномы фиксированного порядка \( k \) (т.е.
степени, не большей \( k \)), et cetera.

\textbf{Линейный оператор} --- отображение между векторными пространствами над
одним полем,
которое ``уважает'' и ``сохраняет'' их линейную структуру, i.e. такое
\( A:X\to Y \), что
\[ A(\alpha x_1 + x_2) = \alpha Ax_1 + Ax_2,\
\alpha\in\mathbb{K}, x_i\in X. \]

\textbf{Алгебра} --- в.п. \( X \) с ассоциативным
умножением \[ (A, B)\mapsto AB: X^2\to X. \]

\textbf{Пространство с внутренним произведением (Inner product space)} --- в.п.,
на котором определено билинейное отображение \( (x,y) \mapsto \langle (x, y) \rangle\)
с естественными свойствами:
\[ \langle  x, y \rangle = \overline{\langle y, x \rangle}, \]
\[ \langle  \alpha x, y \rangle = \alpha \langle x, y \rangle, \]
\[ \langle  x + y, z \rangle =
  \langle x, z \rangle
  + \langle y, z \rangle.
\]

\textbf{Спектр (spectrum) (квадратной) матрицы (эндоморфизма) \( A \)} ---
множество \( \sigma(A) \) таких чисел \( \lambda\in\mathbb{C} \)
для которых матрица \( E - \lambda A \) (оператор \( I - \lambda A \))
необоратима.

\textbf{Собственное подпространство (eigenspace, Eigenraum)
  матрицы \( A \), соответствующее собственному значению \( \lambda \)}
--- множество всех векторов \( x \), удовлетворяющих уравнению \( Ax = \lambda
x\). Оно образует линейное подпространство.

(Ориентированный мульти-)\textbf{Graph} --- тройка \(G = (V, E, \phi)\)
из множества \( V \) вершин, множества \( E \) рёбер,
отображения \( \phi: E\to V^2\).

Граф \textbf{неориентированный (undirected)} если для всех \( e\in E, \phi(e)=(u, v)\)
найдётся \( f\in E, \phi(f)=(v, u) \).

\textbf{Ребро (edge)} --- элемент \( e\in E \); если \( \phi(e) = (u, v)\in V^2 \),
то будем говорить ``ребро \( e \) исходит (originates at) из \( u \) в
(terminates at) \( v \)'', или просто ``ребро из \( u \) в \( v \)''.

Вершина \( v \) называется инцидентной (\textbf{incident with}) с ребром \( e \)
если \( e \) исходит из, и/или входит в \( v \).

Вершины \( u, v\) называются смежными (\textbf{adjacent})
если существует ребро, инцидентное им обоим: \( e\in E \),
\( \phi(e)\in\{(u, v),\ (v, u) \}\).
\newpage
\section{Список литературы}
\nocite{*}
\printbibliography[heading=none,env=gostbibliography]
\end{document}