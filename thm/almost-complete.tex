    Пусть \( M < \frac{1}{16} N^2 \),
    где \( M \) --- число удалённых рёбер, а \( N \) --- число вершин в графе,
    тогда спектр матрицы \( A \) смежностей почти-полного орграфа можно представить в~виде
    объединения \( \sigma\left(A\right) = \sigma_1 \cup \sigma_2 \)
    непересекающихся
    одноэлементного множества \( \sigma_1=\{\lambda_1\} \)
    и~множества \( \sigma_2 \), удовлетворяющих условиям:
    \[ \sigma_1 \subset \left\{ \mu\in\mathbb{R}; \lvert \mu - N \rvert < 4\sqrt{M} \right\}, \]
    \[ \sigma_2 \subset \left\{ \mu\in\mathbb{C}; \lvert \mu \rvert < 4\sqrt{M} \right\}. \]